% Homework template for Inference and Information
% UPDATE: September 26, 2017 by Xiangxiang
\documentclass[a4paper]{article}
\usepackage{ctex}
\usepackage{amsmath, amssymb, amsthm}
\usepackage{moreenum}
\usepackage{mathtools}
\usepackage{url}
\usepackage{bm}
\usepackage{enumitem}
\usepackage{graphicx}
\usepackage{listings}
\usepackage{multirow}
\usepackage{siunitx}
\usepackage{float} % 用于固定图片位置
\usepackage{geometry}

\lstset{
    basicstyle          =   \sffamily,          % 基本代码风格
    keywordstyle        =   \bfseries,          % 关键字风格
    commentstyle        =   \rmfamily\itshape,  % 注释的风格,斜体
    stringstyle         =   \ttfamily,          % 字符串风格
    flexiblecolumns,                
    numbers             =   left,   % 行号的位置在左边
    showspaces          =   false,  % 是否显示空格
    numberstyle         =   \zihao{-5}\ttfamily,    % 行号的样式
    showstringspaces    =   false,
    captionpos          =   t,      % 这段代码的名字所呈现的位置
    frame               =   lrtb,   % 显示边框
}

\lstdefinestyle{Python}{
    language        =   Python, 
    basicstyle      =   \zihao{-5}\ttfamily,
    numberstyle     =   \zihao{-5}\ttfamily,
    keywordstyle    =   \color{blue},
    keywordstyle    =   [2] \color{teal},
    stringstyle     =   \color{magenta},
    commentstyle    =   \color{red}\ttfamily,
    breaklines      =   true,   
    columns         =   fixed,  
    basewidth       =   0.5em,
}

% a4, 0.8
\geometry{a4paper, scale=0.8}

\usepackage[caption=false,font=footnotesize,labelfont=rm,textfont=rm,subrefformat=parens]{subfig}
\usepackage{booktabs} 
\usepackage[mathcal]{eucal}
\usepackage{color}
% \usepackage{iidef} % 注释掉未提供的包,避免编译错误
\newif\ifans\anstrue
\newcommand{\myspace}[1]{\par\vspace{#1\baselineskip}}

% 简单的课程信息定义,替代 iidef 包中的命令
\newcommand{\thecourseinstitute}[1]{\def\courseinstitute{#1}}
\newcommand{\theterm}[1]{\def\term{#1}}
\thecourseinstitute{\textnormal{通信与网络}}
\theterm{2024秋季}

\begin{document}

\vspace{3mm}
\centerline{\textbf{\Large{实验3$\quad$载波传输实验报告}}}

\setcounter{section}{4}

\section{实验内容}

\subsection{BPSK调制}

\begin{enumerate}[label=(\arabic*)]
    \item \textbf{给出前10个符号的发送波形和接收波形图。}
    
    图 \ref{fig:bpsk_wave_tx} 展示了前10个符号的发送波形,其中包含了基带脉冲成形信号与调制后的载波信号。图 \ref{fig:bpsk_wave_rx} 展示了叠加了高斯白噪声后的接收信号波形。
    
    \begin{figure}[H]
        \centering
        \includegraphics[width=0.8\textwidth]{images/exp1-前10个符号波形.png}
        \caption{BPSK前10个符号的发送波形(基带与载波)}
        \label{fig:bpsk_wave_tx}
    \end{figure}

    \begin{figure}[H]
        \centering
        \includegraphics[width=0.8\textwidth]{images/exp1-接收信号波形.png}
        \caption{BPSK接收信号波形(含噪声)}
        \label{fig:bpsk_wave_rx}
    \end{figure}
  
    \item \textbf{画出前5个符号的匹配滤波输出波形(采样前),并在图中明确标出最佳采样时刻。}
    
    图 \ref{fig:bpsk_mf_out} 展示了两种不同匹配滤波方法的输出。
    \begin{itemize}
        \item \textbf{方法1(上图):} 接收信号直接与载波匹配滤波器卷积。输出呈现高频振荡特性,但在最佳采样时刻(红点所示)能取到最大值。
        \item \textbf{方法2(下图):} 接收信号先下变频再进行基带匹配滤波。输出为三角波形状,包络清晰。
    \end{itemize}
    
    \begin{figure}[H]
        \centering
        \includegraphics[width=0.8\textwidth]{images/exp1-前5个符号输出波形.png}
        \caption{BPSK匹配滤波输出及最佳采样时刻}
        \label{fig:bpsk_mf_out}
    \end{figure}

	\item \textbf{分别比较误符号率和误比特率的理论值与仿真值。}
	
	对于BPSK调制,误符号率(SER)与误比特率(BER)相等。图 \ref{fig:bpsk_ber} 和 图 \ref{fig:bpsk_ser} 分别展示了不同采样偏差下的BER和SER性能曲线。
	
	\textbf{分析:}
	\begin{itemize}
	    \item 在无采样偏差(offset 0)的情况下,方法1和方法2的仿真曲线均与理论曲线(红线)高度重合,验证了匹配滤波器的最佳接收性能。
	    \item 当存在采样偏差时,误码率性能均有下降。可以明显观察到,方法1(实线)随着偏差增加,性能恶化极其严重;而方法2(虚线)对采样偏差的鲁棒性更强。
	\end{itemize}

    \begin{figure}[H]
        \centering
        \includegraphics[width=0.7\textwidth]{images/exp1-BPSK误比特率性能曲线.png}
        \caption{BPSK误比特率(BER)性能曲线对比}
        \label{fig:bpsk_ber}
    \end{figure}
    
    \begin{figure}[H]
        \centering
        \includegraphics[width=0.7\textwidth]{images/exp1-BPSK误码率性能曲线.png}
        \caption{BPSK误码率(SER)性能曲线对比}
        \label{fig:bpsk_ser}
    \end{figure}

	\item \textbf{分析误码率随采样偏差变化的规律。}
	
	图 \ref{fig:bpsk_offset} 展示了固定信噪比下,误码率随采样点偏差(-3到3个采样点)的变化情况。
	
	\begin{figure}[H]
        \centering
        \includegraphics[width=0.7\textwidth]{images/exp1-误码率随采样偏差变化.png}
        \caption{误码率随采样偏差的变化规律}
        \label{fig:bpsk_offset}
    \end{figure}
    
    \textbf{深度分析:}
    \begin{itemize}
        \item \textbf{方法1:} 误码率对采样时刻极其敏感,呈深V型。这是因为匹配滤波器的冲激响应包含载波项 $\cos(2\pi f_c t)$。若采样时刻偏差 $\Delta t$,则输出会乘以因子 $\cos(2\pi f_c \Delta t)$。由于载波频率 $f_c=500$Hz 较高,微小的 $\Delta t$ 会导致该余弦项幅值剧烈波动甚至过零,导致信噪比急剧下降。
        \item \textbf{方法2:} 误码率变化较平缓。因为该方法先去除了高频载波,剩下的基带信号通过矩形脉冲匹配滤波后,其自相关函数为三角波。三角波在顶点附近变化缓慢,因此少量的采样偏差(offset 1-2点)仅导致幅值轻微下降,不会引起像方法1那样的剧烈恶化。
    \end{itemize}

	\item \textbf{比较画出的功率谱密度与理论功率谱密度。}
	
	图 \ref{fig:bpsk_psd} 展示了仿真得到的BPSK信号功率谱密度与理论值的对比。
	
	\textbf{分析:}
	仿真结果(蓝色)与理论曲线(红色)吻合良好。功率谱主要集中在载波频率 $f_c = \pm 500$Hz 附近,呈现 $\text{sinc}^2$ 函数形状。主瓣宽度由符号周期 $T$ 决定,约为 $2/T = 200$Hz。

    \begin{figure}[H]
        \centering
        \includegraphics[width=0.7\textwidth]{images/exp1-功率谱.png}
        \caption{BPSK信号功率谱密度(仿真值 vs 理论值)}
        \label{fig:bpsk_psd}
    \end{figure}

\end{enumerate}


\subsection{4PAM调制}

\begin{enumerate}[label=(\arabic*)]
    \item \textbf{给出前5个符号的发送波形和接收波形图。}
    
    图 \ref{fig:pam_wave_tx} 展示了前10个符号的发送波形,其中包含了基带脉冲成形信号与调制后的载波信号。图 \ref{fig:pam_wave_rx} 展示了叠加了高斯白噪声后的接收信号波形。
    
    \begin{figure}[H]
        \centering
        \includegraphics[width=0.8\textwidth]{images/exp2-前5个符号波形.png}
        \caption{4PAM前5个符号的发送波形(基带与载波)}
        \label{fig:pam_wave_tx}
    \end{figure}

    \begin{figure}[H]
        \centering
        \includegraphics[width=0.8\textwidth]{images/exp2-接收信号波形.png}
        \caption{4PAM接收信号波形(含噪声)}
        \label{fig:pam_wave_rx}
    \end{figure}

    \item \textbf{画出前5个符号的匹配滤波输出波形(采样前),并在图中明确标出最佳采样时刻。}
    
    图 \ref{fig:pam_mf_out} 展示了4PAM调制下的匹配滤波输出。可以看到输出电平具有多级特性(对应4PAM的4个符号电平),最佳采样点位于波峰处。
    
    \begin{figure}[H]
        \centering
        \includegraphics[width=0.8\textwidth]{images/exp2-前5个符号输出波形.png}
        \caption{4PAM匹配滤波输出及最佳采样时刻}
        \label{fig:pam_mf_out}
    \end{figure}

	\item \textbf{分别比较误符号率和误比特率的理论值与仿真值。}
	
	图 \ref{fig:pam_ber} 和 图 \ref{fig:pam_ser} 分别展示了4PAM的BER和SER性能。
	
	\textbf{分析:}
	\begin{itemize}
	    \item 在无偏置时,仿真结果与理论值吻合。需要注意的是,4PAM采用了格雷码映射,因此在低误码率下,误比特率约为误符号率的一半($BER \approx SER / \log_2 M = SER / 2$)。
	    \item 与BPSK实验结论一致,方法1对采样偏差非常敏感,而方法2则相对稳健。相比BPSK,4PAM达到相同误码率需要更高的信噪比,因为其星座点更加密集,抗噪性能较差。
	    \end{itemize}

    \begin{figure}[H]
        \centering
        \includegraphics[width=0.7\textwidth]{images/exp2-4PAM误比特率性能曲线.png}
        \caption{4PAM误比特率(BER)性能曲线}
        \label{fig:pam_ber}
    \end{figure}
    
    \begin{figure}[H]
        \centering
        \includegraphics[width=0.7\textwidth]{images/exp2-4PAM误码率性能曲线.png}
        \caption{4PAM误符号率(SER)性能曲线}
        \label{fig:pam_ser}
    \end{figure}

	\item \textbf{比较画出的功率谱密度与理论功率谱密度。}
	
	图 \ref{fig:pam_psd} 为4PAM信号的功率谱。其形状与BPSK相同(均为矩形脉冲成形导致的 $\text{sinc}^2$ 谱),中心频率同样在 $\pm 500$Hz,区别仅在于幅值比例因子不同。
	
	\begin{figure}[H]
        \centering
        \includegraphics[width=0.7\textwidth]{images/exp2-功率谱.png}
        \caption{4PAM信号功率谱密度}
        \label{fig:pam_psd}
    \end{figure}

\end{enumerate}

\subsection{对比BPSK和4PAM的误码曲线}

综合对比图 \ref{fig:bpsk_ber} 和 图 \ref{fig:pam_ber}:
\begin{itemize}
    \item \textbf{频谱效率:} 4PAM在相同的符号速率下传输2 bit信息,而BPSK传输1 bit,因此4PAM的频谱效率是BPSK的2倍。
    \item \textbf{功率效率:} 为了达到相同的误比特率,4PAM需要的 $E_b/N_0$ 显著高于BPSK。这是因为为了保持平均符号能量一致,4PAM的星座点间距比BPSK更小,更容易受到噪声干扰而发生判决错误。
\end{itemize}


\section{实验总结、体会和建议}
\begin{enumerate}
    \item 通过仿真验证了匹配滤波器是高斯白噪声背景下的最佳接收滤波器,能最大化输出信噪比。
    \item 揭示了方法1与方法2在工程实现上的差异。虽然理论性能一致,但方法1对定时误差(采样相位)极其敏感;而方法2由于利用了基带脉冲较宽的相关峰,对定时误差有较强的容忍度,更适合实际系统。
    \item 对比BPSK与4PAM,理解了通信系统中有效性(频谱效率)与可靠性(误码性能)之间的权衡关系。
\end{enumerate}

\end{document}