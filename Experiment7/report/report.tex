% Homework template for Inference and Information
% Modified based on report.tex for Experiment 7
\documentclass[a4paper]{article}
\usepackage{ctex}
\usepackage{amsmath, amssymb, amsthm}
\usepackage{moreenum}
\usepackage{mathtools}
\usepackage{url}
\usepackage{bm}
\usepackage{enumitem}
\usepackage{graphicx}
\usepackage{listings}
\usepackage{multirow}
\usepackage{float}
\usepackage{siunitx}
\usepackage{geometry}
\geometry{a4paper, scale=0.8}

% 代码块设置
\lstset{
    basicstyle          =   \sffamily,          
    keywordstyle        =   \bfseries,          
    commentstyle        =   \rmfamily\itshape,  
    stringstyle         =   \ttfamily,          
    flexiblecolumns,                
    numbers             =   left,   
    showspaces          =   false,  
    numberstyle         =   \zihao{-5}\ttfamily,    
    showstringspaces    =   false,
    captionpos          =   t,      
    frame               =   lrtb,   
}

\usepackage[caption=false,font=footnotesize,labelfont=rm,textfont=rm,subrefformat=parens]{subfig}
\usepackage{booktabs} 
\usepackage[mathcal]{eucal}
\usepackage{color}

% 课程信息配置(如果没有 iidef 包,请注释掉或定义为空)
% \usepackage{iidef} 
\newcommand{\thecourseinstitute}[1]{} % 占位,防止报错
\newcommand{\theterm}[1]{}
%\thecourseinstitute{\textnormal{通信与网络}}
%\theterm{确定}

\begin{document}

\vspace{3mm}
\centerline{\textbf{\Large{实验7$\quad$路由实验报告}}}

\setcounter{section}{4}

\section{实验内容}

\subsection{RIP协议与DV算法仿真}

\begin{enumerate}
    \item 打开实验文件``exp7\_1a.m'',阅读实验代码并运行实验文件,记录Matlab命令行输出的不同迭代轮次下的距离矩阵和下一跳矩阵。选择感兴趣某一次迭代,结合代码对Bellman-Ford算法的实现,对距离矩阵和下一跳矩阵的更新给出解释。
    
    MATLAB输出如下:
    \begin{verbatim}
拓扑节点数: 7,拓扑边数: 11
--------------- LOOP START --------------
距离矢量矩阵:
     0     2     5     1   Inf   Inf   Inf
     2     0     3     2   Inf   Inf   Inf
     5     3     0     3     1     5   Inf
     1     2     3     0     1   Inf   Inf
   Inf   Inf     1     1     0     2   Inf
   Inf   Inf     5   Inf     2     0     1
   Inf   Inf   Inf   Inf   Inf     1     0

下一跳矩阵:
     0     2     3     4     0     0     0
     1     0     3     4     0     0     0
     1     2     0     4     5     6     0
     1     2     3     0     5     0     0
     0     0     3     4     0     6     0
     0     0     3     0     5     0     7
     0     0     0     0     0     6     0

---------------- LOOP 1 ----------------
距离矩阵:
     0     2     4     1     2     9   Inf
     2     0     3     2     3     8   Inf
     3     3     0     2     1     3     6
     1     2     2     0     1     3   Inf
     2     3     1     1     0     2     3
     4     5     3     3     2     0     1
     5     6     4     4     3     1     0

下一跳矩阵:
     0     2     4     4     4     4     0
     1     0     3     4     4     3     0
     5     2     0     5     5     5     6
     1     2     5     0     5     5     0
     4     4     3     4     0     6     6
     5     5     5     5     5     0     7
     6     6     6     6     6     6     0

---------------- LOOP 2 ----------------
距离矩阵:
     0     2     3     1     2     4    10
     2     0     3     2     3     5     9
     3     3     0     2     1     3     4
     1     2     2     0     1     3     4
     2     3     1     1     0     2     3
     4     5     3     3     2     0     1
     5     6     4     4     3     1     0

下一跳矩阵:
     0     2     4     4     4     4     4
     1     0     3     4     4     4     3
     5     2     0     5     5     5     5
     1     2     5     0     5     5     5
     4     4     3     4     0     6     6
     5     5     5     5     5     0     7
     6     6     6     6     6     6     0

---------------- LOOP 3 ----------------
距离矩阵:
     0     2     3     1     2     4     5
     2     0     3     2     3     5     6
     3     3     0     2     1     3     4
     1     2     2     0     1     3     4
     2     3     1     1     0     2     3
     4     5     3     3     2     0     1
     5     6     4     4     3     1     0

下一跳矩阵:
     0     2     4     4     4     4     4
     1     0     3     4     4     4     4
     5     2     0     5     5     5     5
     1     2     5     0     5     5     5
     4     4     3     4     0     6     6
     5     5     5     5     5     0     7
     6     6     6     6     6     6     0

路由收敛,总传输次数 58 次,总传输包大小 306 。
达到循环次数上限或路由收敛,路由信息计算结束。
    \end{verbatim}
    选择第 1 次迭代进行解释:初始时节点只知道直连邻居的距离,不知道非邻居节点。在第一次迭代中,所有节点轮流广播,当节点 d 进行广播时发生更新。节点 d 直连节点 c(代价 3)和节点 e(代价 1)。节点 a 发现经由节点 d 去往节点 c 路径长度由 5 变为 4,于是更新距离矩阵 (a,c) 为 4 和下一跳矩阵 (a,c) 为 d;同时发现经由 d 可以去往 e 且路径长度为 2,更新距离矩阵 (a,e) 为 2 和下一跳矩阵 (a,e) 为 d。节点 e 类似。

    \item 打开实验文件``exp7\_2.m'',阅读实验代码并运行实验文件。在本次实验中,我们在路由表收敛后,将路由节点 6 和 7 之间的链路断开(链路代价无穷大)。运行实验文件,观察 MATLAB 命令行输出的不同迭代轮次下的距离矩阵和下一跳矩阵。基于命令行的输出,观察并记录路由重新收敛的过程。结合 D-V 协议,分析该路由重新收敛的过程中的问题。

    MATLAB 输出如下:
    \begin{verbatim}
拓扑节点数: 7,拓扑边数: 11
--------------- LOOP START --------------
距离矢量矩阵:
     0     2     5     1   Inf   Inf   Inf
     2     0     3     2   Inf   Inf   Inf
     5     3     0     3     1     5   Inf
     1     2     3     0     1   Inf   Inf
   Inf   Inf     1     1     0     2   Inf
   Inf   Inf     5   Inf     2     0     1
   Inf   Inf   Inf   Inf   Inf     1     0

下一跳矩阵:
     0     2     3     4     0     0     0
     1     0     3     4     0     0     0
     1     2     0     4     5     6     0
     1     2     3     0     5     0     0
     0     0     3     4     0     6     0
     0     0     3     0     5     0     7
     0     0     0     0     0     6     0

---------------- LOOP 1 ----------------
距离矩阵:
     0     2     4     1     2     9   Inf
     2     0     3     2     3     8   Inf
     3     3     0     2     1     3     6
     1     2     2     0     1     3   Inf
     2     3     1     1     0     2     3
     4     5     3     3     2     0     1
     5     6     4     4     3     1     0

下一跳矩阵:
     0     2     4     4     4     4     0
     1     0     3     4     4     3     0
     5     2     0     5     5     5     6
     1     2     5     0     5     5     0
     4     4     3     4     0     6     6
     5     5     5     5     5     0     7
     6     6     6     6     6     6     0

---------------- LOOP 2 ----------------
距离矩阵:
     0     2     3     1     2     4    10
     2     0     3     2     3     5     9
     3     3     0     2     1     3     4
     1     2     2     0     1     3     4
     2     3     1     1     0     2     3
     4     5     3     3     2     0     1
     5     6     4     4     3     1     0

下一跳矩阵:
     0     2     4     4     4     4     4
     1     0     3     4     4     4     3
     5     2     0     5     5     5     5
     1     2     5     0     5     5     5
     4     4     3     4     0     6     6
     5     5     5     5     5     0     7
     6     6     6     6     6     6     0

---------------- LOOP 3 ----------------
距离矩阵:
     0     2     3     1     2     4     5
     2     0     3     2     3     5     6
     3     3     0     2     1     3     4
     1     2     2     0     1     3     4
     2     3     1     1     0     2     3
     4     5     3     3     2     0     1
     5     6     4     4     3     1     0

下一跳矩阵:
     0     2     4     4     4     4     4
     1     0     3     4     4     4     4
     5     2     0     5     5     5     5
     1     2     5     0     5     5     5
     4     4     3     4     0     6     6
     5     5     5     5     5     0     7
     6     6     6     6     6     6     0

路由收敛,总传输次数 58 次,总传输包大小 306 。
达到循环次数上限或路由收敛,路由信息计算结束。

6-7链路断开...

重新计算路由...
--------------- LOOP START --------------
距离矢量矩阵:
     0     2     3     1     2     4     5
     2     0     3     2     3     5     6
     3     3     0     2     1     3     4
     1     2     2     0     1     3     4
     2     3     1     1     0     2     3
     4     5     3     3     2     0   Inf
   Inf   Inf   Inf   Inf   Inf   Inf     0

下一跳矩阵:
     0     2     4     4     4     4     4
     1     0     3     4     4     4     4
     5     2     0     5     5     5     5
     1     2     5     0     5     5     5
     4     4     3     4     0     6     6
     5     5     5     5     5     0     0
     0     0     0     0     0     0     0

---------------- LOOP 1 ----------------
距离矩阵:
     0     2     3     1     2     4     5
     2     0     3     2     3     5     6
     3     3     0     2     1     3     4
     1     2     2     0     1     3     4
     2     3     1     1     0     2     7
     4     5     3     3     2     0     5
   Inf   Inf   Inf   Inf   Inf   Inf     0

下一跳矩阵:
     0     2     4     4     4     4     4
     1     0     3     4     4     4     4
     5     2     0     5     5     5     5
     1     2     5     0     5     5     5
     4     4     3     4     0     6     6
     5     5     5     5     5     0     5
     0     0     0     0     0     0     0

---------------- LOOP 2 ----------------
距离矩阵:
     0     2     3     1     2     4     5
     2     0     3     2     3     5     6
     3     3     0     2     1     3     6
     1     2     2     0     1     3     6
     2     3     1     1     0     2     5
     4     5     3     3     2     0     7
   Inf   Inf   Inf   Inf   Inf   Inf     0

下一跳矩阵:
     0     2     4     4     4     4     4
     1     0     3     4     4     4     4
     5     2     0     5     5     5     5
     1     2     5     0     5     5     5
     4     4     3     4     0     6     3
     5     5     5     5     5     0     5
     0     0     0     0     0     0     0

---------------- LOOP 3 ----------------
距离矩阵:
     0     2     3     1     2     4     7
     2     0     3     2     3     5     8
     3     3     0     2     1     3     8
     1     2     2     0     1     3     8
     2     3     1     1     0     2     7
     4     5     3     3     2     0     9
   Inf   Inf   Inf   Inf   Inf   Inf     0

下一跳矩阵:
     0     2     4     4     4     4     4
     1     0     3     4     4     4     4
     5     2     0     5     5     5     5
     1     2     5     0     5     5     5
     4     4     3     4     0     6     3
     5     5     5     5     5     0     5
     0     0     0     0     0     0     0

---------------- LOOP 4 ----------------
距离矩阵:
     0     2     3     1     2     4     9
     2     0     3     2     3     5    10
     3     3     0     2     1     3    10
     1     2     2     0     1     3    10
     2     3     1     1     0     2     9
     4     5     3     3     2     0    11
   Inf   Inf   Inf   Inf   Inf   Inf     0

下一跳矩阵:
     0     2     4     4     4     4     4
     1     0     3     4     4     4     4
     5     2     0     5     5     5     5
     1     2     5     0     5     5     5
     4     4     3     4     0     6     3
     5     5     5     5     5     0     5
     0     0     0     0     0     0     0

---------------- LOOP 5 ----------------
距离矩阵:
     0     2     3     1     2     4    11
     2     0     3     2     3     5    12
     3     3     0     2     1     3    12
     1     2     2     0     1     3    12
     2     3     1     1     0     2    11
     4     5     3     3     2     0    13
   Inf   Inf   Inf   Inf   Inf   Inf     0

下一跳矩阵:
     0     2     4     4     4     4     4
     1     0     3     4     4     4     4
     5     2     0     5     5     5     5
     1     2     5     0     5     5     5
     4     4     3     4     0     6     3
     5     5     5     5     5     0     5
     0     0     0     0     0     0     0

---------------- LOOP 6 ----------------
距离矩阵:
     0     2     3     1     2     4    13
     2     0     3     2     3     5    14
     3     3     0     2     1     3    14
     1     2     2     0     1     3    14
     2     3     1     1     0     2    13
     4     5     3     3     2     0    15
   Inf   Inf   Inf   Inf   Inf   Inf     0

下一跳矩阵:
     0     2     4     4     4     4     4
     1     0     3     4     4     4     4
     5     2     0     5     5     5     5
     1     2     5     0     5     5     5
     4     4     3     4     0     6     3
     5     5     5     5     5     0     5
     0     0     0     0     0     0     0

---------------- LOOP 7 ----------------
距离矩阵:
     0     2     3     1     2     4    15
     2     0     3     2     3     5    16
     3     3     0     2     1     3    16
     1     2     2     0     1     3    16
     2     3     1     1     0     2    15
     4     5     3     3     2     0    17
   Inf   Inf   Inf   Inf   Inf   Inf     0

下一跳矩阵:
     0     2     4     4     4     4     4
     1     0     3     4     4     4     4
     5     2     0     5     5     5     5
     1     2     5     0     5     5     5
     4     4     3     4     0     6     3
     5     5     5     5     5     0     5
     0     0     0     0     0     0     0

---------------- LOOP 8 ----------------
距离矩阵:
     0     2     3     1     2     4    17
     2     0     3     2     3     5    18
     3     3     0     2     1     3    18
     1     2     2     0     1     3    18
     2     3     1     1     0     2    17
     4     5     3     3     2     0    19
   Inf   Inf   Inf   Inf   Inf   Inf     0

下一跳矩阵:
     0     2     4     4     4     4     4
     1     0     3     4     4     4     4
     5     2     0     5     5     5     5
     1     2     5     0     5     5     5
     4     4     3     4     0     6     3
     5     5     5     5     5     0     5
     0     0     0     0     0     0     0

---------------- LOOP 9 ----------------
距离矩阵:
     0     2     3     1     2     4    19
     2     0     3     2     3     5    20
     3     3     0     2     1     3    20
     1     2     2     0     1     3    20
     2     3     1     1     0     2    19
     4     5     3     3     2     0    21
   Inf   Inf   Inf   Inf   Inf   Inf     0

下一跳矩阵:
     0     2     4     4     4     4     4
     1     0     3     4     4     4     4
     5     2     0     5     5     5     5
     1     2     5     0     5     5     5
     4     4     3     4     0     6     3
     5     5     5     5     5     0     5
     0     0     0     0     0     0     0

---------------- LOOP 10 ----------------
距离矩阵:
     0     2     3     1     2     4    21
     2     0     3     2     3     5    22
     3     3     0     2     1     3    22
     1     2     2     0     1     3    22
     2     3     1     1     0     2    21
     4     5     3     3     2     0    23
   Inf   Inf   Inf   Inf   Inf   Inf     0

下一跳矩阵:
     0     2     4     4     4     4     4
     1     0     3     4     4     4     4
     5     2     0     5     5     5     5
     1     2     5     0     5     5     5
     4     4     3     4     0     6     3
     5     5     5     5     5     0     5
     0     0     0     0     0     0     0

---------------- LOOP 11 ----------------
距离矩阵:
     0     2     3     1     2     4    23
     2     0     3     2     3     5    24
     3     3     0     2     1     3    24
     1     2     2     0     1     3    24
     2     3     1     1     0     2    23
     4     5     3     3     2     0    25
   Inf   Inf   Inf   Inf   Inf   Inf     0

下一跳矩阵:
     0     2     4     4     4     4     4
     1     0     3     4     4     4     4
     5     2     0     5     5     5     5
     1     2     5     0     5     5     5
     4     4     3     4     0     6     3
     5     5     5     5     5     0     5
     0     0     0     0     0     0     0

---------------- LOOP 12 ----------------
距离矩阵:
     0     2     3     1     2     4    25
     2     0     3     2     3     5    26
     3     3     0     2     1     3    26
     1     2     2     0     1     3    26
     2     3     1     1     0     2    25
     4     5     3     3     2     0    27
   Inf   Inf   Inf   Inf   Inf   Inf     0

下一跳矩阵:
     0     2     4     4     4     4     4
     1     0     3     4     4     4     4
     5     2     0     5     5     5     5
     1     2     5     0     5     5     5
     4     4     3     4     0     6     3
     5     5     5     5     5     0     5
     0     0     0     0     0     0     0

---------------- LOOP 13 ----------------
距离矩阵:
     0     2     3     1     2     4    27
     2     0     3     2     3     5    28
     3     3     0     2     1     3    28
     1     2     2     0     1     3    28
     2     3     1     1     0     2    27
     4     5     3     3     2     0    29
   Inf   Inf   Inf   Inf   Inf   Inf     0

下一跳矩阵:
     0     2     4     4     4     4     4
     1     0     3     4     4     4     4
     5     2     0     5     5     5     5
     1     2     5     0     5     5     5
     4     4     3     4     0     6     3
     5     5     5     5     5     0     5
     0     0     0     0     0     0     0

---------------- LOOP 14 ----------------
距离矩阵:
     0     2     3     1     2     4    29
     2     0     3     2     3     5    30
     3     3     0     2     1     3    30
     1     2     2     0     1     3    30
     2     3     1     1     0     2    29
     4     5     3     3     2     0    31
   Inf   Inf   Inf   Inf   Inf   Inf     0

下一跳矩阵:
     0     2     4     4     4     4     4
     1     0     3     4     4     4     4
     5     2     0     5     5     5     5
     1     2     5     0     5     5     5
     4     4     3     4     0     6     3
     5     5     5     5     5     0     5
     0     0     0     0     0     0     0

---------------- LOOP 15 ----------------
距离矩阵:
     0     2     3     1     2     4    31
     2     0     3     2     3     5    32
     3     3     0     2     1     3    32
     1     2     2     0     1     3    32
     2     3     1     1     0     2    31
     4     5     3     3     2     0    33
   Inf   Inf   Inf   Inf   Inf   Inf     0

下一跳矩阵:
     0     2     4     4     4     4     4
     1     0     3     4     4     4     4
     5     2     0     5     5     5     5
     1     2     5     0     5     5     5
     4     4     3     4     0     6     3
     5     5     5     5     5     0     5
     0     0     0     0     0     0     0

---------------- LOOP 16 ----------------
距离矩阵:
     0     2     3     1     2     4    33
     2     0     3     2     3     5    34
     3     3     0     2     1     3    34
     1     2     2     0     1     3    34
     2     3     1     1     0     2    33
     4     5     3     3     2     0    35
   Inf   Inf   Inf   Inf   Inf   Inf     0

下一跳矩阵:
     0     2     4     4     4     4     4
     1     0     3     4     4     4     4
     5     2     0     5     5     5     5
     1     2     5     0     5     5     5
     4     4     3     4     0     6     3
     5     5     5     5     5     0     5
     0     0     0     0     0     0     0

达到循环次数上限或路由收敛,路由信息计算结束。
    \end{verbatim}
    迭代不收敛。当 6-7 链路断开后,节点 6 失去了去往 7 的直连路径。但在下一次更新前,其邻居(如节点 5)的路由表中仍然保留着旧的路由信息。随后节点 5 发现 6 到 7 的距离变大了,也随之更新自己的路由表,导致两者距离交替上升,无法收敛。
  
\end{enumerate}


\subsection{OSPF协议与LS算法 (exp7\_3.m)}
打开实验文件``exp7\_3.m'',阅读实验代码,结合实验代码理解链路状态协议。补齐第8行实验代码,通过将变量``viewNode''设置为 1 到 N 之间的一个整数,选中某一路由节点,观察该路由节点的最小生成树随链路状态传播而变化的情况。运行实验文件,观察 MATLAB 命令行输出和生成的图片。观察并记录,链路状态协议下链路状态的泛洪过程。记录生成的图片中,``viewNode''路由节点的最小生成树随链路状态传播而变化的情况,结合 Dijkstra 算法,给出解释。

选中节点 1,MATLAB 命令行输出如下:
\begin{verbatim}
完成第1次泛洪迭代...

MST =

     0     2     0     1     0     0     0
     2     0     0     0     0     0     0
     0     0     0     0     1     5     0
     1     0     0     0     1     0     0
     0     0     1     1     0     0     0
     0     0     5     0     0     0     0
     0     0     0     0     0     0     0

完成第2次泛洪迭代...

MST =

     0     2     0     1     0     0     0
     2     0     0     0     0     0     0
     0     0     0     0     1     0     0
     1     0     0     0     1     0     0
     0     0     1     1     0     2     0
     0     0     0     0     2     0     1
     0     0     0     0     0     1     0

完成第3次泛洪迭代...

MST =

     0     2     0     1     0     0     0
     2     0     0     0     0     0     0
     0     0     0     0     1     0     0
     1     0     0     0     1     0     0
     0     0     1     1     0     2     0
     0     0     0     0     2     0     1
     0     0     0     0     0     1     0

完成第4次泛洪迭代...

MST =

     0     2     0     1     0     0     0
     2     0     0     0     0     0     0
     0     0     0     0     1     0     0
     1     0     0     0     1     0     0
     0     0     1     1     0     2     0
     0     0     0     0     2     0     1
     0     0     0     0     0     1     0

路由收敛,总传输次数 112 次,总传输包大小 240。
\end{verbatim}

选中节点 1,它在不同泛洪轮次下的 MST 如图 \ref{fig:ospf_mst} 所示。

泛洪过程:
\begin{enumerate}
    \item 节点 1 收到节点 2、3、4发来的 LSA。此时节点 1 仅掌握邻居的连接情况。
    \item 邻居节点转发了它们收到的 LSA。节点 1 此时收到了节点 5、6 发出的 LSA.
    \item 节点 1 收到节点 7 的 LSA。
    \item 网络中所有 LSA 已传遍所有节点,没有新的 LSA 需要转发。
\end{enumerate}

MST 矩阵变化:
\begin{enumerate}
    \item MST 矩阵显示节点 a 与 b、d 建立了连接,并通过 d 连接到 e,通过 e 连接到 c,通过 c 连接到 f。但节点 g 的行/列全为 0,表示节点 a 此时认为节点 g 不可达。
    \item 邻居节点 b、c、d 转发了它们上一轮收到的 LSA,节点 g 通过节点 f 连接到 MST.
    \item MST 矩阵不再发生变化,路由表已收敛。
\end{enumerate}

\begin{figure}[H]
    \centering
    \subfloat[第1轮泛洪]{
        \includegraphics[width=0.45\linewidth]{images/MST1_1.png}}
    \hfill
    \subfloat[第2轮泛洪]{
        \includegraphics[width=0.45\linewidth]{images/MST1_2.png}}
    \\
    \subfloat[第3轮泛洪]{
        \includegraphics[width=0.45\linewidth]{images/MST1_3.png}}
    \hfill
    \subfloat[第4轮泛洪]{
        \includegraphics[width=0.45\linewidth]{images/MST1_4.png}}
    \caption{节点 1 的最小生成树}
    \label{fig:ospf_mst}
\end{figure}



\section{实验总结}
通过 Bellman-Ford 距离矢量算法和 OSPF 链路状态算法对路由更新进行了仿真,发现了 Bellman-Ford 算法在链路故障时容易产生路由环路和计数到无穷的问题,且收敛速度较慢。

\end{document}