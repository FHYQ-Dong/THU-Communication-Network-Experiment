% Homework template for Inference and Information
% UPDATE: September 26, 2017 by Xiangxiang
\documentclass[a4paper]{article}
\usepackage{ctex}
\usepackage{amsmath, amssymb, amsthm}
\usepackage{moreenum}
\usepackage{mathtools}
\usepackage{url}
\usepackage{bm}
\usepackage{enumitem}
\usepackage{graphicx}
\usepackage{listings}
\usepackage{multirow}
\usepackage{float}
\usepackage{siunitx}
\lstset{
    basicstyle          =   \sffamily,          % 基本代码风格
    keywordstyle        =   \bfseries,          % 关键字风格
    commentstyle        =   \rmfamily\itshape,  % 注释的风格,斜体
    stringstyle         =   \ttfamily,  % 字符串风格
    flexiblecolumns,                % 
    numbers             =   left,   % 行号的位置在左边
    showspaces          =   false,  % 是否显示空格,显示了有点乱,所以不显示了
    numberstyle         =   \zihao{-5}\ttfamily,    % 行号的样式,小五号,tt等宽字体
    showstringspaces    =   false,
    captionpos          =   t,      % 这段代码的名字所呈现的位置,t指的是top上面
    frame               =   lrtb,   % 显示边框
}

\lstdefinestyle{Python}{
    language        =   Python, % 语言选Python
    basicstyle      =   \zihao{-5}\ttfamily,
    numberstyle     =   \zihao{-5}\ttfamily,
    keywordstyle    =   \color{blue},
    keywordstyle    =   [2] \color{teal},
    stringstyle     =   \color{magenta},
    commentstyle    =   \color{red}\ttfamily,
    breaklines      =   true,   % 自动换行,建议不要写太长的行
    columns         =   fixed,  % 如果不加这一句,字间距就不固定,很丑,必须加
    basewidth       =   0.5em,
}
% \usepackage{subcaption}
\usepackage[caption=false,font=footnotesize,labelfont=rm,textfont=rm,subrefformat=parens]{subfig}
\usepackage{booktabs} % toprule
\usepackage[mathcal]{eucal}
\usepackage{color}
\usepackage{iidef}
\newif\ifans\anstrue
\newcommand{\myspace}[1]{\par\vspace{#1\baselineskip}}

\thecourseinstitute{\textnormal{通信与网络}}
\theterm{确定}
\begin{document}



\vspace{3mm}
\centerline{\textbf{\Large{实验5$\quad$频分多址实验报告}}}

\setcounter{section}{4}

\section{实验内容}
% \subsection{BPSK调制}

\begin{enumerate}[label=(\arabic*)]
    \item 绘制发射信号波形功率谱。根据功率谱密度计算各用户的功率和总功率。
    
    功率谱密度如图 \ref{fig:psd} 所示。
    \begin{figure}[H]
        \centering
        \includegraphics[width=0.7\textwidth]{images/Power_Spectrum_SNR.png}
        \caption{发射信号功率谱密度}
        \label{fig:psd}
    \end{figure}
    程序输出总功率和各用户功率如下:
    \begin{verbatim}
总功率 400 W
用户1:中心频率 200 Hz,功率 45.8653 W
用户2:中心频率 400 Hz,功率 49.0588 W
用户3:中心频率 600 Hz,功率 49.2762 W
用户4:中心频率 800 Hz,功率 48.086 W
    \end{verbatim}

    \item 在多个噪声功率谱密度下,比较发送信号和接收信号,统计误符号率、误比特率。画出误比特率与 $E_b/n_0$ 之间的关系曲线,并与理论误比特率曲线进行对比。
    
    扫描 $E_b/n_0$ 范围为 \lstinline|-5:1:5|. 由于是BPSK调制,误符号率与误比特率相等。误比特率与 $E_b/n_0$ 曲线如图 \ref{fig:ber} 所示,各个 $E_b/n_0$ 情况下的发送与接收波形图如图 \ref{fig:wave} 所示。

    \begin{figure}[H]
        \centering
        \includegraphics[width=0.7\textwidth]{images/BER_Curve_Comparison.png}
        \caption{误比特率与 $E_b/n_0$ 曲线}
        \label{fig:ber}
    \end{figure}

    图中将四个用户的误比特率进行了平均。经验证,各个用户之间的误比特率差异不超过 $1\%$,平均是合理的。仿真结果与理论值一致。

    \begin{figure}[H]
        \centering
        % 第一行: -5dB, -4dB, -3dB
        \subfloat[$E_b/n_0 = -5\text{ dB}$]{
            \includegraphics[width=0.32\linewidth]{images/Waveform_SNR_-5dB.png}}
        \hfill
        \subfloat[$E_b/n_0 = -4\text{ dB}$]{
            \includegraphics[width=0.32\linewidth]{images/Waveform_SNR_-4dB.png}}
        \hfill
        \subfloat[$E_b/n_0 = -3\text{ dB}$]{
            \includegraphics[width=0.32\linewidth]{images/Waveform_SNR_-3dB.png}}
        \\ % 换行
        
        % 第二行: -2dB, -1dB, 0dB
        \subfloat[$E_b/n_0 = -2\text{ dB}$]{
            \includegraphics[width=0.32\linewidth]{images/Waveform_SNR_-2dB.png}}
        \hfill
        \subfloat[$E_b/n_0 = -1\text{ dB}$]{
            \includegraphics[width=0.32\linewidth]{images/Waveform_SNR_-1dB.png}}
        \hfill
        \subfloat[$E_b/n_0 = 0\text{ dB}$]{
            \includegraphics[width=0.32\linewidth]{images/Waveform_SNR_0dB.png}}
        \\ % 换行

        % 第三行: 1dB, 2dB, 3dB
        \subfloat[$E_b/n_0 = 1\text{ dB}$]{
            \includegraphics[width=0.32\linewidth]{images/Waveform_SNR_1dB.png}}
        \hfill
        \subfloat[$E_b/n_0 = 2\text{ dB}$]{
            \includegraphics[width=0.32\linewidth]{images/Waveform_SNR_2dB.png}}
        \hfill
        \subfloat[$E_b/n_0 = 3\text{ dB}$]{
            \includegraphics[width=0.32\linewidth]{images/Waveform_SNR_3dB.png}}
        \\ % 换行

        % 第四行: 4dB, 5dB (居中排列)
        \subfloat[$E_b/n_0 = 4\text{ dB}$]{
            \includegraphics[width=0.32\linewidth]{images/Waveform_SNR_4dB.png}}
        \hspace{0.02\linewidth} % 微调间距
        \subfloat[$E_b/n_0 = 5\text{ dB}$]{
            \includegraphics[width=0.32\linewidth]{images/Waveform_SNR_5dB.png}}
        
        \caption{不同 $E_b/n_0$ 下的发送与接收波形对比}
        \label{fig:wave}
    \end{figure}

\end{enumerate}


\section{实验总结、体会和建议}
    \begin{itemize}
        \item 实现了基于 BPSK 调制的基带信号产生、矩形脉冲成形、载波调制、AWGN 信道传输以及相干解调与最佳接收判决的全过程。
        \item 利用滑动窗 FFT 方法绘制了发射信号的功率谱密度图像,验证了频分复用的频谱特性。通过对功率谱密度的积分,计算出的总功率与各用户功率与理论预期基本吻合。
        \item 在 $E_b/n_0$ 从 \SI{-5}{\dB} 到 \SI{5}{\dB} 的范围内进行了仿真。系统的误比特率曲线与 BPSK 理论曲线高度重合。
    \end{itemize}
\end{document}




%%% Local Variables:
%%% mode: late\rvx
%%% TeX-master: t
%%% End:
