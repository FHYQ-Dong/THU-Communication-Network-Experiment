% Homework template for Inference and Information
% UPDATE: September 26, 2017 by Xiangxiang
\documentclass[a4paper]{article}
\usepackage{ctex}
\usepackage{amsmath, amssymb, amsthm}
\usepackage{moreenum}
\usepackage{mathtools}
\usepackage{url}
\usepackage{bm}
\usepackage{enumitem}
\usepackage{graphicx}
\usepackage{listings}
\usepackage{multirow}
\usepackage{siunitx}
\usepackage[caption=false,font=footnotesize,labelfont=rm,textfont=rm,subrefformat=parens]{subfig}
\usepackage{booktabs} 
\usepackage[mathcal]{eucal}
\usepackage{color}
\usepackage{iidef} % 如果编译报错缺少此包,请确保同目录下有iidef.sty,或者注释掉
\usepackage{float} % 强制图片位置
\usepackage{geometry}

\newif\ifans\anstrue
\newcommand{\myspace}[1]{\par\vspace{#1\baselineskip}}

\geometry{a4paper, scale=0.8}



\begin{document}

\vspace{3mm}
\centerline{\textbf{\Large{实验6$\quad$ALOHA}}} 

\setcounter{section}{3} % 调整章节编号以匹配指导书

\section{实验内容与结果分析}

\begin{enumerate}[label=(\arabic*)]
    \item 设置用户数 $N = 50$,单用户的帧到达率 $\lambda$ 范围为 \lstinline|1:1:60|. 分别绘制出纯ALOHA和Slotted ALOHA的理论计算的 $\rho-G$ 曲线和在上述两种帧到达过程设置下的仿真 $\rho-G$ 曲线。观察结果,并解释仿真结果和理论结果有差距的原因。
    
    如图 \ref{fig:pure_aloha_all} 和图 \ref{fig:slotted_aloha_all} 所示,观察仿真图像 $N=50$,在设置1(纯泊松过程)下,仿真得到的 $\rho-G$ 曲线与理论公式相符。在设置2(非修正 $\lambda$)下,随着网络负载 $G$ 的增加,仿真得到的吞吐率 $\rho$ 明显高于理论值。

    原因:理论公式假设帧的到达服从泊松过程,即用户可以在任意时刻产生新帧,包括上一帧正在发送的期间。而在设置2中,模拟了更符合实际的场景:用户在发送某一帧的持续时间 $T_{fr}$ 内无法产生新的数据帧。这种机制导致了实际的帧生成间隔变大。当名义上的参数 $\lambda$ 不变时,由于加入了固定的 $T_{fr}$ 间隔,用户的实际平均帧到达率更低。因此,对于相同的横坐标,实际注入网络的流量更小,冲突较少。因此表现出的吞吐率 $\rho$ 更高。

    \item 在(1)的绘图结果的基础上多绘制一条仿真 $\rho-G$ 曲线:在帧到达过程设置2下修正 $\lambda$ 使之更接近于真实的到达率,观察结果。代码中给出了一种修正方法,解释它的原理。
    
    如图 \ref{fig:pure_aloha_all} 和图 \ref{fig:slotted_aloha_all} 所示,使用修正后的 $\lambda_{real}$ 进行仿真后,得到的 $\rho-G$ 曲线与理论曲重新重合。
    
    原理:在设置2中,一个完整的发送周期由两部分组成:
    \begin{itemize}
        \item 数据帧发送时间:固定为 $T_{fr}$。
        \item 下一帧到达前的等待时间:服从均值为 $1/\lambda$ 的指数分布。
    \end{itemize}
    因此,一个用户发送两帧之间的平均时间间隔 $E[T_{interval}]$ 为:
    \begin{equation}
        E[T_{interval}] = T_{fr} + E[T_{idle}] = T_{fr} + \frac{1}{\lambda}
    \end{equation}
    用户的真实到达率 $\lambda_{real}$ 为:
    \begin{equation}
        \lambda_{real} = \frac{1}{E[T_{interval}]} = \frac{1}{T_{fr} + \frac{1}{\lambda}} = \frac{\lambda}{1 + \lambda T_{fr}}
    \end{equation}
    这即是代码中使用的修正公式。

    \item 改变用户数 $N$ ,重做(2),观察总结 $N$ 对仿真结果的影响,并解释。
    
    扫描 $N$ 的范围为 \lstinline|20:10:80|,结果如图 \ref{fig:pure_aloha_all} 和图 \ref{fig:slotted_aloha_all} 所示。

    随着 $N$ 的增加,设置2中未修正的仿真点与理论曲线的差距逐渐缩小。
    
    原因:考查修正因子 $\frac{\lambda_{real}}{\lambda} = \frac{1}{1 + \lambda T_{fr}}$,为了保持归一化总负载 $G = N \lambda T_{fr}$ 为一个固定的常数,当用户数 $N$ 增大时,单用户的到达率 $\lambda$ 必须相应减小。
    \begin{itemize}
        \item 当 $N$ 较小时,单用户 $\lambda$ 较大,$\lambda T_{fr}$ 较大,分母 $1+\lambda T_{fr}$ 显著大于1,修正效应明显,因此未修正前的误差很大。
        \item 当 $N$ 较大时,单用户 $\lambda$ 很小,$\lambda T_{fr} \to 0$,分母 $1+\lambda T_{fr} \to 1$,即 $\lambda_{real} \approx \lambda$。此时到达过程更接近纯泊松过程,因此偏差减小。
    \end{itemize}
    综上所述,用户数越多,单个用户的负载越轻,系统的行为就越接近理想的泊松过程假设。
    
    \begin{figure}[H]
        \centering
        \subfloat[N=20]{\includegraphics[width=0.32\textwidth]{images/ALOHA_N_20.png}}
        \hfill
        \subfloat[N=30]{\includegraphics[width=0.32\textwidth]{images/ALOHA_N_30.png}}
        \hfill
        \subfloat[N=40]{\includegraphics[width=0.32\textwidth]{images/ALOHA_N_40.png}}
        \\
        \subfloat[N=50]{\includegraphics[width=0.32\textwidth]{images/ALOHA_N_50.png}}
        \hfill
        \subfloat[N=60]{\includegraphics[width=0.32\textwidth]{images/ALOHA_N_60.png}}
        \hfill
        \subfloat[N=70]{\includegraphics[width=0.32\textwidth]{images/ALOHA_N_70.png}}
        \\
        \subfloat[N=80]{\includegraphics[width=0.32\textwidth]{images/ALOHA_N_80.png}}
        \caption{不同用户数 $N$ 下的纯 ALOHA 协议仿真对比}
        \label{fig:pure_aloha_all}
    \end{figure}

    \begin{figure}[H]
        \centering
        \subfloat[N=20]{\includegraphics[width=0.32\textwidth]{images/Slotted_ALOHA_N_20.png}}
        \hfill
        \subfloat[N=30]{\includegraphics[width=0.32\textwidth]{images/Slotted_ALOHA_N_30.png}}
        \hfill
        \subfloat[N=40]{\includegraphics[width=0.32\textwidth]{images/Slotted_ALOHA_N_40.png}}
        \\
        \subfloat[N=50]{\includegraphics[width=0.32\textwidth]{images/Slotted_ALOHA_N_50.png}}
        \hfill
        \subfloat[N=60]{\includegraphics[width=0.32\textwidth]{images/Slotted_ALOHA_N_60.png}}
        \hfill
        \subfloat[N=70]{\includegraphics[width=0.32\textwidth]{images/Slotted_ALOHA_N_70.png}}
        \\
        \subfloat[N=80]{\includegraphics[width=0.32\textwidth]{images/Slotted_ALOHA_N_80.png}}
        \caption{不同用户数 $N$ 下的 Slotted ALOHA 协议仿真对比}
        \label{fig:slotted_aloha_all}
    \end{figure}

\end{enumerate}


\section{实验总结、体会和建议}
实验验证了纯 ALOHA 最大吞吐率为 $1/2e \approx 0.184$,而 Slotted ALOHA 通过时隙同步将脆弱时间减半,使最大吞吐率提升至 $1/e \approx 0.368$。此外,通过对比两种不同的帧到达设置,我认识到理论模型(泊松过程)与实际系统(存在发送阻塞)之间的差异。

\end{document}
