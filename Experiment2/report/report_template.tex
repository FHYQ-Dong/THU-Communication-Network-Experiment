% Homework template for Inference and Information
% UPDATE: September 26, 2017 by Xiangxiang
\documentclass[a4paper]{article}
\usepackage{ctex}
\usepackage{amsmath, amssymb, amsthm}
\usepackage{moreenum}
\usepackage{mathtools}
\usepackage{url}
\usepackage{bm}
\usepackage{enumitem}
\usepackage{graphicx}
\usepackage{listings}
\usepackage{multirow}
\usepackage{siunitx}
\lstset{
    basicstyle          =   \sffamily,          % 基本代码风格
    keywordstyle        =   \bfseries,          % 关键字风格
    commentstyle        =   \rmfamily\itshape,  % 注释的风格,斜体
    stringstyle         =   \ttfamily,  % 字符串风格
    flexiblecolumns,                % 
    numbers             =   left,   % 行号的位置在左边
    showspaces          =   false,  % 是否显示空格,显示了有点乱,所以不显示了
    numberstyle         =   \zihao{-5}\ttfamily,    % 行号的样式,小五号,tt等宽字体
    showstringspaces    =   false,
    captionpos          =   t,      % 这段代码的名字所呈现的位置,t指的是top上面
    frame               =   lrtb,   % 显示边框
}

\lstdefinestyle{Python}{
    language        =   Python, % 语言选Python
    basicstyle      =   \zihao{-5}\ttfamily,
    numberstyle     =   \zihao{-5}\ttfamily,
    keywordstyle    =   \color{blue},
    keywordstyle    =   [2] \color{teal},
    stringstyle     =   \color{magenta},
    commentstyle    =   \color{red}\ttfamily,
    breaklines      =   true,   % 自动换行,建议不要写太长的行
    columns         =   fixed,  % 如果不加这一句,字间距就不固定,很丑,必须加
    basewidth       =   0.5em,
}
% \usepackage{subcaption}
\usepackage[caption=false,font=footnotesize,labelfont=rm,textfont=rm,subrefformat=parens]{subfig}
\usepackage{booktabs} % toprule
\usepackage[mathcal]{eucal}
\usepackage{color}
\usepackage{iidef}
\newif\ifans\anstrue
\newcommand{\myspace}[1]{\par\vspace{#1\baselineskip}}

\thecourseinstitute{\textnormal{通信与网络}}
\theterm{确定}
\begin{document}



\vspace{3mm}
\centerline{\textbf{\Large{实验2$\quad$基带传输实验报告}}}

\setcounter{section}{4}

\section{实验内容}
\subsection{实电平信道传输}

\begin{enumerate}[label=(\arabic*)]
    \item 在合适的信噪比范围下生成对应的高斯噪声,与信源符号相加,得到接收信号。高斯噪声的方差可以由平均符号能量和信噪比计算得到。其中,平均符号能量有两种取法,分别是:
    \begin{enumerate}
        \item 随机符号序列的平均能量;
        \item 星座图的理论符号能量。
    \end{enumerate}
    请选择其中一种,并说明原因。
    \item 比较原始比特序列和解调后的比特序列。分别统计不同信噪比、不同仿真次数$N$下的符号差错个数和比特差错个数。数据记录表可以参考表\ref{tab:symbol}和表\ref{tab:bit}。请自行选取有代表性的仿真次数$N$和信噪比SNR。
    \begin{table*}[h]
        \setlength{\tabcolsep}{20pt}
        \centering
        \caption{示例-符号差错个数统计,$M=2$}
        \vspace{10pt}
        \label{tab:symbol}
        \begin{tabular}{|c|c|c|c|}
            \hline
            \multirow{2}{*}{SNR} & \multicolumn{3}{c|}{仿真次数$N$}\\
            \cline{2-4}
             & $10^4$ & $10^5$ & $10^6$ \\
            \hline
            2 &  &  &  \\
            \hline
            5 &  &  &  \\
            \hline
            8 &  &  &  \\
            \hline
        \end{tabular}
    \end{table*}

    \begin{table*}[h]
        \setlength{\tabcolsep}{20pt}
        \centering
        \caption{示例-bit差错个数统计,$M=4$}
        \vspace{10pt}
        \label{tab:bit}
        \begin{tabular}{|c|c|c|c|}
            \hline
            \multirow{2}{*}{SNR} & \multicolumn{3}{c|}{仿真次数$N$}\\
            \cline{2-4}
            & $10^4$ & $10^5$ & $10^6$ \\
            \hline
            8 &  &  &  \\
            \hline
            11 &  &  &  \\
            \hline
            14 &  &  &  \\
            \hline
        \end{tabular}
    \end{table*}
    \item 在合适的仿真次数下,计算误符号率$P_s$和误比特率$P_b$。分别画出$M=2$和$M=4$时,$P_s$和$P_b$随信噪比$E_s/\sigma^2$的变化曲线。一般情况下,当独立的错误事件数量达到100量级时,计算得到的误码率就比较稳定了。其中,要求仿真曲线涉及的SNR范围包括误码率从$10^{-3}$到$10^{-1}$。
    \item 同时画出理论值对应的曲线,观察仿真结果是否与理论值相符,分析原因。
\end{enumerate}


\subsection{复电平信道传输}

\begin{enumerate}[label=(\arabic*)]
	\item 对于$M=16$的QAM调制情况,请选择两个合适的信噪比条件画出接收信号的复电平星座图。
	\item 比较发送信号和接收信号。分别统计不同信噪比、不同仿真次数$N$下的符号差错个数和bit差错个数。数据记录表可以参考表\ref{tab:symbol}和表\ref{tab:bit}。请自行选取有代表性的仿真次数$N$和信噪比SNR。
	\item 在合适的仿真次数下,计算误符号率$P_s$和误比特率$P_b$。分别画出$M=4$和$M=16$时,$P_s$和$P_b$随信噪比$\text{SNR}=E_s/\sigma^2$的变化曲线。其中,要求仿真曲线涉及的SNR范围包括误码率从$10^{-3}$到$10^{-1}$。
	\item 同时画出理论值对应的曲线,观察仿真结果是否与理论值相符,分析原因。
\end{enumerate}


\subsection{波形信道传输}

\begin{enumerate}[label=(\arabic*)]
	\item 对于两个不同的采样时间间隔$\Delta t$,选择合适的信噪比画出接收端有噪声波形在一个符号持续时间内的波形,即对于时间长度为$T$的符号波形$y_t$的$T/\Delta t$个采样电平。结合观察说明为什么对接收波形直接采样判决不妥。
    \item 分别画出$\Delta t=0.1$和$\Delta t=0.01$时,等效信噪比随波形持续时间$T$的变化曲线。
    \item 分别画出$\Delta t=0.1$和$\Delta t=0.01$时,误符号率随信噪比的变化曲线。同时画出理论值对应的曲线,观察仿真结果是否与理论值相符,以及两种$\Delta t$下结果的区别,分析原因。
    \item 为什么随着持续时间$T$的提升,等效信噪比增加且误码率降低?
\end{enumerate}

\section{实验总结、体会和建议}

\end{document}




%%% Local Variables:
%%% mode: late\rvx
%%% TeX-master: t
%%% End:
